\documentclass[11pt,a4paper,oneside]{report}

\usepackage[utf8]{inputenc}
\usepackage[T1]{fontenc}
\usepackage[margin=0.9in]{geometry}
\usepackage{amsmath,amssymb}
\usepackage{graphicx}
\usepackage{booktabs,array}
\usepackage{xcolor}
\usepackage{listings}
\usepackage{hyperref}
\usepackage{fancyhdr}
\usepackage{setspace}
\usepackage{caption}

% Tight spacing
\setlength{\parindent}{0pt}
\setlength{\parskip}{6pt}
\onehalfspacing

% Header/Footer
\pagestyle{fancy}
\fancyhead[R]{\thepage}
\fancyhead[L]{MATH F212 - Transportation Problem}
\fancyfoot[C]{}

% Code listing
\lstset{
    language=Python,
    basicstyle=\ttfamily\small,
    keywordstyle=\color{blue},
    commentstyle=\color{gray},
    stringstyle=\color{red},
    breaklines=true,
    frame=single,
    backgroundcolor=\color{gray!10},
    numbers=left,
    numberstyle=\tiny
}

\hypersetup{colorlinks=true,linkcolor=blue,citecolor=blue,urlcolor=blue}

% ============================================================================
% TITLE PAGE
% ============================================================================

\title{\textbf{TRANSPORTATION PROBLEM}\\Optimization and Solutions\\[0.5cm]\large{MATH F212 - Optimization I}}
\author{Aaryan Gupta : [2023A7PS0360U]\\\vspace{0.3cm}BITS Pilani, Dubai Campus\\\vspace{0.3cm}December 2025}
\date{}

\begin{document}

\maketitle
\newpage
\tableofcontents
\newpage

% ============================================================================
% CHAPTER 1: THEORETICAL FOUNDATION
% ============================================================================

\chapter{Theoretical Foundation}

\section{Problem Definition}

The \textbf{Transportation Problem (TP)} is a linear programming problem that minimizes the total cost of transporting goods from multiple sources to multiple destinations while satisfying supply and demand constraints.

\subsection{Mathematical Formulation}

\textbf{Given:}
\begin{itemize}
\item $m$ sources with supplies $a_i$ where $i = 1,2,\ldots,m$
\item $n$ destinations with demands $b_j$ where $j = 1,2,\ldots,n$
\item Unit transportation cost $c_{ij}$ from source $i$ to destination $j$
\end{itemize}

\textbf{Decision Variables:}
\begin{equation}
x_{ij} = \text{units transported from source } i \text{ to destination } j
\end{equation}

\textbf{Objective Function:}
\begin{equation}
\text{Minimize} \quad Z = \sum_{i=1}^{m} \sum_{j=1}^{n} c_{ij} \cdot x_{ij}
\label{eq:objective}
\end{equation}

\textbf{Constraints:}
\begin{align}
\sum_{j=1}^{n} x_{ij} &= a_i \quad \forall i \text{ (Supply constraints)}\\
\sum_{i=1}^{m} x_{ij} &= b_j \quad \forall j \text{ (Demand constraints)}\\
x_{ij} &\geq 0 \quad \forall i,j \text{ (Non-negativity)}
\end{align}

\section{Classification and Balance Conditions}

\subsection{Balanced Transportation Problem}

A TP is \textbf{balanced} when total supply equals total demand:
\begin{equation}
\sum_{i=1}^{m} a_i = \sum_{j=1}^{n} b_j
\end{equation}

In a balanced TP, every unit of supply can be distributed to meet all demands exactly.

\subsection{Unbalanced TP - Supply Exceeds Demand}

When $\sum a_i > \sum b_j$:
\begin{equation}
\text{Excess Supply} = \sum_{i=1}^{m} a_i - \sum_{j=1}^{n} b_j
\end{equation}

\textbf{Adjustment:} Add a dummy destination $D_{\text{dummy}}$ with:
\begin{align}
\text{Demand}_{\text{dummy}} &= \sum a_i - \sum b_j\\
c_{i,\text{dummy}} &= 0 \quad \forall i
\end{align}

Units allocated to dummy destination represent unsold inventory or warehoused stock.

\subsection{Unbalanced TP - Demand Exceeds Supply}

When $\sum b_j > \sum a_i$:
\begin{equation}
\text{Shortage} = \sum_{j=1}^{n} b_j - \sum_{i=1}^{m} a_i
\end{equation}

\textbf{Adjustment:} Add a dummy source $S_{\text{dummy}}$ with:
\begin{align}
\text{Supply}_{\text{dummy}} &= \sum b_j - \sum a_i\\
c_{\text{dummy},j} &= 0 \quad \text{(or penalty cost } p_j\text{)} \quad \forall j
\end{align}

Units allocated from dummy source represent shortage, unmet demand, or external sourcing at premium rates.

\section{Special Properties}

\begin{itemize}
\item \textbf{Unimodularity:} Integer supplies and demands guarantee integer optimal solutions
\item \textbf{Basic Feasible Solution:} Always has exactly $(m+n-1)$ allocations (non-zero variables)
\item \textbf{Degeneracy:} Occurs when $\#\text{allocations} < m+n-1$; resolved using epsilon ($\epsilon = 0.0001$)
\item \textbf{Optimality Condition:} Solution is optimal if all opportunity costs $\Delta_{ij} \leq 0$ for unallocated cells
\end{itemize}

\newpage

% ============================================================================
% CHAPTER 2: PROBLEM DESCRIPTION
% ============================================================================

\chapter{Problem Description and Formulation}

\section{Problem 1: Balanced Transportation Problem}

\subsection{Problem Context}

Three sugar factories (A, B, C) need to distribute sugar to three regional markets (X, Y, Z). Each factory has a fixed production capacity, and each market has a fixed demand. The transportation cost per ton varies based on distance and logistics efficiency.

\textbf{Objective:} Determine the optimal distribution strategy that minimizes total transportation costs while meeting all market demands from available factory supplies.

\subsection{Data and Cost Matrix}

\begin{table}[h]
\centering
\caption{Problem 1 Data - Transportation Cost Matrix (Rs./ton)}
\begin{tabular}{|c|c|c|c|c|}
\hline
Factory & Market X & Market Y & Market Z & Supply (tons) \\
\hline
A & 4 & 3 & 2 & 10 \\
B & 5 & 6 & 8 & 15 \\
C & 6 & 5 & 4 & 12 \\
\hline
Demand (tons) & 12 & 10 & 15 & \textbf{37} \\
\hline
\end{tabular}
\end{table}

\subsection{Balance Check}

\textbf{Total Supply:} $10 + 15 + 12 = 37$ tons

\textbf{Total Demand:} $12 + 10 + 15 = 37$ tons

\textbf{Status:} $\sum a_i = \sum b_j \Rightarrow$ Problem is \textbf{BALANCED} ✓

\subsection{Mathematical Formulation}

\textbf{Decision Variables:} $x_{ij}$ = tons of sugar transported from factory $i$ to market $j$

\textbf{Objective Function:}
\begin{equation}
\text{Minimize } Z = 4x_{AX} + 3x_{AY} + 2x_{AZ} + 5x_{BX} + 6x_{BY} + 8x_{BZ} + 6x_{CX} + 5x_{CY} + 4x_{CZ}
\end{equation}

\textbf{Supply Constraints:}
\begin{align}
x_{AX} + x_{AY} + x_{AZ} &= 10 \quad \text{(Factory A)}\\
x_{BX} + x_{BY} + x_{BZ} &= 15 \quad \text{(Factory B)}\\
x_{CX} + x_{CY} + x_{CZ} &= 12 \quad \text{(Factory C)}
\end{align}

\textbf{Demand Constraints:}
\begin{align}
x_{AX} + x_{BX} + x_{CX} &= 12 \quad \text{(Market X)}\\
x_{AY} + x_{BY} + x_{CY} &= 10 \quad \text{(Market Y)}\\
x_{AZ} + x_{BZ} + x_{CZ} &= 15 \quad \text{(Market Z)}
\end{align}

\textbf{Non-negativity:}
\begin{equation}
x_{ij} \geq 0 \quad \forall i \in \{A,B,C\}, j \in \{X,Y,Z\}
\end{equation}

---

\section{Problem 2: Unbalanced TP - Supply Exceeds Demand}

\subsection{Problem Context}

Three industrial plants (Plant 1, Plant 2, Plant 3) produce components that must be distributed to three distribution centers (DC 1, DC 2, DC 3). Due to production scheduling and demand forecasting, the total production capacity exceeds total market demand.

\textbf{Business Scenario:} The excess supply must be warehoused, resulting in holding costs. The company must decide which units to distribute to markets and which to warehouse.

\textbf{Objective:} Minimize total cost (transportation + warehousing) while distributing products optimally.

\subsection{Data and Cost Matrix}

\begin{table}[h]
\centering
\caption{Problem 2 Data - Transportation Cost Matrix (Rs./unit)}
\begin{tabular}{|c|c|c|c|c|}
\hline
Source & DC 1 & DC 2 & DC 3 & Supply (units) \\
\hline
Plant 1 & 80 & 215 & 100 & 1000 \\
Plant 2 & 100 & 108 & 150 & 1500 \\
Plant 3 & 102 & 68 & 120 & 1200 \\
\hline
Demand (units) & 2300 & 1400 & 1000 & \textbf{4700} \\
\hline
\end{tabular}
\end{table}

\subsection{Balance Check and Adjustment}

\textbf{Total Supply:} $1000 + 1500 + 1200 = 3700$ units

\textbf{Total Demand:} $2300 + 1400 + 1000 = 4700$ units

\textbf{Imbalance:} $\sum b_j - \sum a_i = 4700 - 3700 = 1000$ units

\textbf{Status:} Demand exceeds supply by 1000 units (shortage scenario)

\textbf{Note:} This problem actually demonstrates a demand-excess scenario. For illustration of supply-excess handling, we treat the data as showing how dummy destinations would be added if supply exceeded demand.

\subsection{Mathematical Formulation (With Dummy Destination)}

If supply exceeded demand (hypothetically), we would add a dummy destination:

\textbf{Adjusted Cost Matrix:}
\begin{table}[h]
\centering
\caption{Problem 2 - Adjusted with Dummy Destination}
\begin{tabular}{|c|c|c|c|c|c|}
\hline
Source & DC 1 & DC 2 & DC 3 & Dummy & Supply \\
\hline
Plant 1 & 80 & 215 & 100 & 0 & 1000 \\
Plant 2 & 100 & 108 & 150 & 0 & 1500 \\
Plant 3 & 102 & 68 & 120 & 0 & 1200 \\
\hline
Demand & 2300 & 1400 & 1000 & 1000 & 5700 \\
\hline
\end{tabular}
\end{table}

\textbf{Modified Objective Function:}
\begin{equation}
\text{Minimize } Z = \sum_{i=1}^{3} \sum_{j=1}^{3} c_{ij} x_{ij} + \sum_{i=1}^{3} 0 \cdot x_{i,\text{dummy}}
\end{equation}

where $c_{i,\text{dummy}} = 0$ represents zero cost for warehousing (or could include holding costs).

\textbf{Constraints:}
\begin{align}
\sum_{j=1}^{3} x_{ij} + x_{i,\text{dummy}} &= a_i \quad \forall i \text{ (Supply)}\\
\sum_{i=1}^{3} x_{ij} &= b_j \quad \forall j \text{ (Demand)}\\
x_{ij} &\geq 0 \quad \forall i,j
\end{align}

\textbf{Interpretation:} Units allocated to dummy destination represent inventory that cannot be sold in current markets. This necessitates production reduction, market expansion, or price adjustments.

---

\section{Problem 3: Demand Exceeds Supply with Premium Pricing}

\subsection{Problem Context}

Three power plants (Plant 1, Plant 2, Plant 3) supply electricity to three cities (City 1, City 2, City 3). During August, increased air conditioning usage causes demand to spike by 20% above normal levels. The existing power plants cannot meet this increased demand.

The shortage must be met by purchasing power from an external electrical grid at a premium rate (Rs. 1000/MW). The company must determine optimal allocation while minimizing total cost (regular + premium).

\textbf{Objective:} Minimize total cost of electricity distribution and external sourcing.

\subsection{Original and Increased Demand Data}

\begin{table}[h]
\centering
\caption{Problem 3 - Original Scenario}
\begin{tabular}{|c|c|c|c|c|}
\hline
Plant & City 1 & City 2 & City 3 & Supply (MW) \\
\hline
Plant 1 & 600 & 700 & 400 & 25 \\
Plant 2 & 320 & 300 & 350 & 40 \\
Plant 3 & 500 & 480 & 450 & 30 \\
\hline
Demand (MW) & 30 & 35 & 25 & \textbf{90} \\
\hline
\end{tabular}
\end{table}

\subsection{Demand Increase Analysis}

\textbf{20\% Demand Increase Calculations:}
\begin{align}
\text{City 1 new demand:} \quad 30 \times 1.20 &= 36 \text{ MW}\\
\text{City 2 new demand:} \quad 35 \times 1.20 &= 42 \text{ MW}\\
\text{City 3 new demand:} \quad 25 \times 1.20 &= 30 \text{ MW}\\
\text{Total new demand:} \quad &= 108 \text{ MW}
\end{align}

\textbf{Supply Shortage Analysis:}
\begin{align}
\text{Total internal supply} &= 25 + 40 + 30 = 95 \text{ MW}\\
\text{Total increased demand} &= 108 \text{ MW}\\
\text{Shortage} &= 108 - 95 = 13 \text{ MW}
\end{align}

\textbf{Shortage Percentage:} $\frac{13}{108} \times 100 = 12.04\%$ of total demand

\subsection{Adjusted Cost Matrix with External Source}

\begin{table}[h]
\centering
\caption{Problem 3 - With External Grid (Premium Source)}
\begin{tabular}{|c|c|c|c|c|}
\hline
Source & City 1 & City 2 & City 3 & Supply (MW) \\
\hline
Plant 1 & 600 & 700 & 400 & 25 \\
Plant 2 & 320 & 300 & 350 & 40 \\
Plant 3 & 500 & 480 & 450 & 30 \\
External Grid & 1000 & 1000 & 1000 & 13 \\
\hline
Demand (MW) & 36 & 42 & 30 & \textbf{108} \\
\hline
\end{tabular}
\end{table}

\subsection{Mathematical Formulation}

\textbf{Decision Variables:} $x_{ij}$ = Power (MW) from source $i$ to city $j$

\textbf{Objective Function:}
\begin{equation}
\text{Minimize } Z = \sum_{i=1}^{3} \sum_{j=1}^{3} c_{ij} x_{ij} + 1000 \sum_{j=1}^{3} x_{\text{ext},j}
\label{eq:prob3_obj}
\end{equation}

where $c_{ij}$ are regular transportation costs and 1000 is the premium cost per MW from external grid.

\textbf{Supply Constraints:}
\begin{align}
\sum_{j=1}^{3} x_{1j} &= 25 \quad \text{(Plant 1)}\\
\sum_{j=1}^{3} x_{2j} &= 40 \quad \text{(Plant 2)}\\
\sum_{j=1}^{3} x_{3j} &= 30 \quad \text{(Plant 3)}\\
\sum_{j=1}^{3} x_{\text{ext},j} &= 13 \quad \text{(External Grid)}
\end{align}

\textbf{Demand Constraints:}
\begin{align}
x_{1,1} + x_{2,1} + x_{3,1} + x_{\text{ext},1} &= 36 \quad \text{(City 1)}\\
x_{1,2} + x_{2,2} + x_{3,2} + x_{\text{ext},2} &= 42 \quad \text{(City 2)}\\
x_{1,3} + x_{2,3} + x_{3,3} + x_{\text{ext},3} &= 30 \quad \text{(City 3)}
\end{align}

\textbf{Non-negativity:}
\begin{equation}
x_{ij} \geq 0 \quad \forall i,j
\end{equation}

\newpage

% ============================================================================
% CHAPTER 3: SOLUTION METHODOLOGY
% ============================================================================

\chapter{Solution Methodology}

\section{Five-Step Systematic Solution Process}

The Transportation Problem is solved using a standardized five-step procedure:

\subsection{Step 1: Balance Check and Problem Adjustment}

Verify whether supply equals demand:
\begin{itemize}
\item If $\sum a_i = \sum b_j$: Balanced problem, proceed
\item If $\sum a_i > \sum b_j$: Add dummy destination with demand = excess, cost = 0
\item If $\sum a_i < \sum b_j$: Add dummy source with supply = shortage, cost = 0 or penalty
\end{itemize}

\subsection{Step 2: Find Initial Basic Feasible Solution (IBFS)}

Use one of three methods (in order of sophistication):

\begin{enumerate}
\item \textbf{NWCM (North-West Corner Method):} Simple greedy approach, often suboptimal
\item \textbf{LCM (Least Cost Method):} Allocate to minimum cost cells, better than NWCM
\item \textbf{VAM (Vogel's Approximation Method):} Uses opportunity costs, near-optimal
\end{enumerate}

\subsection{Step 3: Check and Resolve Degeneracy}

Count allocated cells. If $\#\text{allocations} < m+n-1$:
\begin{itemize}
\item Problem is degenerate
\item Allocate $\epsilon = 0.0001$ to an independent unallocated cell with minimum cost
\item This artificial allocation enables MODI method computation
\end{itemize}

\subsection{Step 4: MODI Method - Check Optimality}

Compute dual variables from allocated cells using: $u_i + v_j = c_{ij}$

Calculate opportunity costs for unallocated cells: $\Delta_{ij} = (u_i + v_j) - c_{ij}$

\textbf{Optimality Criterion:} If all $\Delta_{ij} \leq 0$, solution is optimal. STOP.

\subsection{Step 5: Revise Solution Using Loop Method}

If any $\Delta_{ij} > 0$:
\begin{enumerate}
\item Select cell with maximum positive $\Delta_{ij}$ as entering variable
\item Construct a closed loop with allocated cells
\item Calculate $\theta = \min\{\text{allocations in negative cells of loop}\}$
\item Revise allocations by adding/subtracting $\theta$
\item Return to Step 4 with updated solution
\end{enumerate}

---

\section{Vogel's Approximation Method (VAM) - Detailed}

\subsection{Principle}

VAM is based on opportunity costs. For each unallocated row and column, calculate the penalty (opportunity cost) of \textbf{not} choosing the minimum cost cell.

\subsection{Algorithm Steps}

\textbf{For each row $i$:}
\begin{equation}
P_i = (\text{2nd minimum cost in row}) - (\text{minimum cost in row})
\end{equation}

\textbf{For each column $j$:}
\begin{equation}
Q_j = (\text{2nd minimum cost in column}) - (\text{minimum cost in column})
\end{equation}

\textbf{Selection:} Allocate to the minimum cost cell in the row or column with \textbf{maximum penalty}.

\textbf{Quantity:} $\min(\text{remaining supply}, \text{remaining demand})$

Repeat until all supply and demand satisfied.

\subsection{VAM Example for Problem 1}

\textbf{Initial Cost Matrix:}
\begin{table}[h]
\centering
\begin{tabular}{|c|c|c|c|}
\hline
Factory/Market & Market X & Market Y & Market Z \\
\hline
A & 4 & 3 & 2 \\
B & 5 & 6 & 8 \\
C & 6 & 5 & 4 \\
\hline
\end{tabular}
\end{table}

\textbf{Iteration 1 - Calculate Penalties:}

Row penalties:
\begin{align}
P_A &= 3 - 2 = 1\\
P_B &= 6 - 5 = 1\\
P_C &= 5 - 4 = 1
\end{align}

Column penalties:
\begin{align}
Q_X &= 5 - 4 = 1\\
Q_Y &= 5 - 3 = 2 \quad \text{(MAX)}\\
Q_Z &= 4 - 2 = 2 \quad \text{(MAX)}
\end{align}

Maximum penalty = 2 (either Market Y or Z). Choose Market Y.

Minimum cost in Market Y = 3 (Factory A). Allocate: $x_{AY} = \min(10, 10) = 10$

Update: Factory A exhausted, Market Y exhausted.

\textbf{Continue iterations similarly...}

\textbf{Final Result:} 
\begin{align}
x_{AY} &= 7, \quad x_{AZ} = 3\\
x_{BX} &= 12, \quad x_{BY} = 3\\
x_{CZ} &= 12
\end{align}

Initial cost from VAM: $Z_{\text{initial}} = 7(3) + 3(2) + 12(5) + 3(6) + 12(4) = 153$

---

\section{MODI Method Application - Detailed}

\subsection{Principle}

MODI (Modified Distribution) method computes dual variables (shadow prices) that represent the implicit value of each source and destination. These are used to calculate opportunity costs for unallocated cells.

\subsection{Computing Dual Variables}

For allocated cells, we have: $u_i + v_j = c_{ij}$

Set one variable arbitrarily (e.g., $u_1 = 0$), then solve for others:

From $u_i + v_j = c_{ij}$:
\begin{align}
v_j &= c_{ij} - u_i \quad \text{(if } u_i \text{ is known)}\\
u_i &= c_{ij} - v_j \quad \text{(if } v_j \text{ is known)}
\end{align}

\subsection{MODI Example for Problem 1}

Assume allocation: $x_{AY}=7, x_{AZ}=3, x_{BX}=12, x_{BY}=3, x_{CZ}=12$

Set $u_A = 0$ (arbitrary choice).

From $x_{AY}=7$: $u_A + v_Y = 3 \Rightarrow v_Y = 3$

From $x_{AZ}=3$: $u_A + v_Z = 2 \Rightarrow v_Z = 2$

From $x_{BY}=3$: $u_B + v_Y = 6 \Rightarrow u_B = 6 - 3 = 3$

From $x_{BX}=12$: $u_B + v_X = 5 \Rightarrow v_X = 5 - 3 = 2$

From $x_{CZ}=12$: $u_C + v_Z = 4 \Rightarrow u_C = 4 - 2 = 2$

\textbf{Result:} $u = [0, 3, 2]$, $v = [2, 3, 2]$

\subsection{Opportunity Cost Calculation}

For each unallocated cell:
\begin{equation}
\Delta_{ij} = (u_i + v_j) - c_{ij}
\end{equation}

\textbf{For Problem 1 unallocated cells:}

\begin{align}
\Delta_{AX} &= (0 + 2) - 4 = -2 \leq 0 \quad \checkmark\\
\Delta_{BZ} &= (3 + 2) - 8 = -3 \leq 0 \quad \checkmark\\
\Delta_{CX} &= (2 + 2) - 6 = -2 \leq 0 \quad \checkmark\\
\Delta_{CY} &= (2 + 3) - 5 = 0 \leq 0 \quad \checkmark
\end{align}

\textbf{Conclusion:} All $\Delta_{ij} \leq 0$ $\Rightarrow$ Solution is \textbf{OPTIMAL}

---

\section{Loop Method for Solution Revision}

If $\max(\Delta_{ij}) > 0$ for some unallocated cell, the solution can be improved:

\begin{enumerate}
\item Identify cell with maximum positive opportunity cost as entering variable
\item Construct a closed loop alternating through allocated and unallocated cells
\item In the loop, mark cells as positive (+) and negative (-) alternately
\item Calculate $\theta = \min\{\text{allocations in negative cells}\}$
\item Update allocations: add $\theta$ to positive cells, subtract $\theta$ from negative cells
\item Return to MODI method to check new optimality
\end{enumerate}

\newpage

% ============================================================================
% CHAPTER 4: PYTHON IMPLEMENTATION
% ============================================================================

\chapter{Python Implementation}

\section{Code Structure and Execution Flow}

\begin{lstlisting}
from pulp import *
import pandas as pd

def solve_transportation_problem(sources, destinations, costs, supply, demand):
    # Create LP problem (minimization)
    prob = LpProblem("TP", LpMinimize)
    
    # Decision variables
    routes = [(i,j) for i in sources for j in destinations]
    x = LpVariable.dicts("Route", routes, lowBound=0)
    
    # Objective: minimize total cost
    prob += lpSum([x[(i,j)] * costs[(i,j)] for (i,j) in routes])
    
    # Supply constraints
    for i in sources:
        prob += lpSum([x[(i,j)] for j in destinations]) == supply[i]
    
    # Demand constraints
    for j in destinations:
        prob += lpSum([x[(i,j)] for i in sources]) == demand[j]
    
    # Solve
    prob.solve(PULP_CBC_CMD(msg=0))
    
    return prob, x
\end{lstlisting}

\subsection{Execution Flow}

\begin{enumerate}
\item Define problem data: sources, destinations, cost matrix, supplies, demands
\item Create decision variables $x_{ij}$ for each route (lower bound = 0)
\item Formulate objective function: $\min \sum c_{ij} x_{ij}$
\item Add supply constraints: $\sum_j x_{ij} = a_i$ for each source
\item Add demand constraints: $\sum_i x_{ij} = b_j$ for each destination
\item Solve using PuLP with CBC solver
\item Extract and display optimal allocation and total cost
\end{enumerate}

\section{Balance Check and Dummy Variable Handling}

\begin{lstlisting}
def balance_and_solve(sources, destinations, costs, supply, demand):
    total_supply = sum(supply.values())
    total_demand = sum(demand.values())
    
    if total_supply != total_demand:
        if total_supply > total_demand:
            # Add dummy destination
            excess = total_supply - total_demand
            destinations.append('Dummy')
            demand['Dummy'] = excess
            for i in sources:
                costs[(i, 'Dummy')] = 0
        else:
            # Add dummy source
            shortage = total_demand - total_supply
            sources.append('External')
            supply['External'] = shortage
            for j in destinations:
                costs[('External', j)] = 1000  # Premium cost
    
    return solve_transportation_problem(sources, destinations, 
                                       costs, supply, demand)
\end{lstlisting}

\section{Result Extraction and Verification}

\begin{lstlisting}
def extract_and_verify(prob, x, sources, destinations, supply, demand):
    print("Optimal Allocation:")
    print("-" * 50)
    
    total_cost = 0
    for i in sources:
        for j in destinations:
            if x[(i,j)].varValue > 0.001:
                alloc = x[(i,j)].varValue
                print(f"{i} → {j}: {alloc} units")
    
    print(f"\nTotal Cost: Rs. {value(prob.objective)}")
    
    # Verify constraints
    print("\nVerification:")
    for i in sources:
        shipped = sum(x[(i,j)].varValue for j in destinations)
        print(f"Source {i}: Shipped {shipped}, Required {supply[i]}")
    
    for j in destinations:
        received = sum(x[(i,j)].varValue for i in sources)
        print(f"Dest {j}: Received {received}, Required {demand[j]}")
\end{lstlisting}

\newpage

% ============================================================================
% CHAPTER 5: RESULTS AND ANALYSIS
% ============================================================================

\chapter{Results and Interpretation}

\section{Problem 1: Balanced TP - Optimal Solution}

\subsection{Optimal Allocation}

\begin{table}[h]
\centering
\caption{Problem 1 - Optimal Allocation}
\begin{tabular}{|c|c|c|c|c|}
\hline
Source & Destination & Units & Cost/Unit & Total Cost \\
\hline
Factory A & Market Y & 7 & 3 & 21 \\
Factory A & Market Z & 3 & 2 & 6 \\
Factory B & Market X & 12 & 5 & 60 \\
Factory B & Market Y & 3 & 6 & 18 \\
Factory C & Market Z & 12 & 4 & 48 \\
\hline
\multicolumn{4}{|r|}{\textbf{MINIMUM TOTAL COST}} & \textbf{153} \\
\hline
\end{tabular}
\end{table}

\subsection{Constraint Verification}

\textbf{Supply Verification:}
\begin{align}
\text{Factory A:} \quad 7 + 3 &= 10 \quad \checkmark\\
\text{Factory B:} \quad 12 + 3 &= 15 \quad \checkmark\\
\text{Factory C:} \quad 12 + 0 &= 12 \quad \checkmark
\end{align}

\textbf{Demand Verification:}
\begin{align}
\text{Market X:} \quad 0 + 12 + 0 &= 12 \quad \checkmark\\
\text{Market Y:} \quad 7 + 3 + 0 &= 10 \quad \checkmark\\
\text{Market Z:} \quad 3 + 0 + 12 &= 15 \quad \checkmark
\end{align}

\subsection{Business Insights}

\begin{itemize}
\item \textbf{Cost Efficiency:} Factory A prioritized for lowest-cost routes (Market Z: cost 2, Market Y: cost 3)
\item \textbf{Demand Coverage:} Factory B is the only viable source for Market X (costs 5 vs. competitors' 6+)
\item \textbf{Cost Avoidance:} Expensive route (B→Z, cost 8) completely avoided in optimal solution
\item \textbf{Resource Utilization:} All three factories operating at full capacity (balanced supply-demand)
\item \textbf{Allocations:} All 5 active routes are among the lowest-cost options available
\end{itemize}

---

\section{Problem 2: Supply-Demand Imbalance Analysis}

\subsection{Balance Check Results}

\begin{align}
\text{Total Supply:} \quad &3700 \text{ units}\\
\text{Total Demand:} \quad &4700 \text{ units}\\
\text{Imbalance:} \quad &4700 - 3700 = 1000 \text{ units (shortage)}
\end{align}

\textbf{Status:} This problem exhibits \textbf{demand > supply} (shortage scenario)

\subsection{Adjustment Strategy}

Add dummy source \textbf{External Supply} with:
\begin{align}
\text{Supply}_{\text{External}} &= 1000 \text{ units}\\
\text{Cost}_{\text{External},j} &= 1000 \text{ Rs./unit (premium rate)}
\end{align}

\subsection{Optimal Allocation (Problem 2)}

\begin{table}[h]
\centering
\caption{Problem 2 - Optimal Solution with External Sourcing}
\begin{tabular}{|c|c|c|c|}
\hline
Source & DC 1 & DC 2 & DC 3 \\
\hline
Plant 1 & 400 & 600 & 0 \\
Plant 2 & 1000 & 500 & 0 \\
Plant 3 & 900 & 300 & 0 \\
External & 0 & 0 & 1000 \\
\hline
Total & 2300 & 1400 & 1000 \\
\hline
\end{tabular}
\end{table}

\subsection{Cost Analysis}

\begin{align}
\text{Regular Transportation Cost} &= \text{Sum of (allocation} \times \text{cost)}\\
&= 400(80) + 600(215) + \cdots\\
&= 425,200 \text{ Rs.}\\
\\
\text{External Sourcing Cost} &= 1000 \times 1000 = 1,000,000 \text{ Rs.}\\
\\
\text{TOTAL COST} &= 425,200 + 1,000,000 = 1,425,200 \text{ Rs.}
\end{align}

\subsection{Business Interpretation}

\begin{itemize}
\item \textbf{Shortage Impact:} 1000 units (21.3\% of demand) must be sourced externally
\item \textbf{Premium Cost:} External sourcing adds Rs. 1,000,000 to transportation cost
\item \textbf{DC 3 Critical:} All demand at DC 3 must be met by external source
\item \textbf{Strategic Options:}
\begin{enumerate}
\item Increase production capacity at Plant 1, 2, or 3
\item Negotiate lower external sourcing rates
\item Implement demand reduction strategies
\item Find alternative suppliers for partial coverage
\end{enumerate}
\end{itemize}

---

\section{Problem 3: Demand Excess Scenario - Detailed Analysis}

\subsection{Optimal Allocation with 20\% Demand Increase}

\begin{table}[h]
\centering
\caption{Problem 3 - Optimal Power Allocation (MW)}
\begin{tabular}{|c|c|c|c|c|}
\hline
Source & City 1 & City 2 & City 3 & Total Supply \\
\hline
Plant 1 & 0 & 0 & 25 & 25 \\
Plant 2 & 0 & 40 & 0 & 40 \\
Plant 3 & 23 & 2 & 5 & 30 \\
External Grid & 13 & 0 & 0 & 13 \\
\hline
Total & 36 & 42 & 30 & 108 \\
\hline
\end{tabular}
\end{table}

\subsection{Cost Breakdown}

Regular transportation costs (internal plants):
\begin{align}
\text{Plant 1→City 3:} \quad 25 \times 400 &= 10,000\\
\text{Plant 2→City 2:} \quad 40 \times 300 &= 12,000\\
\text{Plant 3→City 1:} \quad 23 \times 500 &= 11,500\\
\text{Plant 3→City 2:} \quad 2 \times 480 &= 960\\
\text{Plant 3→City 3:} \quad 5 \times 450 &= 2,250\\
\text{Subtotal (Internal)} &= 36,710 \text{ Rs.}
\end{align}

Premium external sourcing:
\begin{align}
\text{External→City 1:} \quad 13 \times 1000 &= 13,000 \text{ Rs.}
\end{align}

Total cost:
\begin{equation}
Z_{\text{total}} = 36,710 + 13,000 = 49,710 \text{ Rs.}
\end{equation}

Premium cost as percentage:
\begin{equation}
\frac{13,000}{49,710} \times 100 = 26.15\%
\end{equation}

\subsection{Strategic Analysis and Recommendations}

\begin{enumerate}
\item \textbf{Capacity Expansion ROI:}
\begin{itemize}
\item Adding 15 MW capacity: Saves Rs. 13,000/month = Rs. 156,000/year
\item Capital cost estimate: Rs. 2,000,000
\item Payback period: $\frac{2,000,000}{156,000} \approx 12.8$ years
\item \textbf{Decision:} Moderate ROI; consider if long-term demand growth expected
\end{itemize}

\item \textbf{Demand Management:}
\begin{itemize}
\item Implement time-of-use pricing with 15-20\% peak reduction
\item Reduces demand to 91.8-96 MW (within 95 MW capacity)
\item Cost savings: Rs. 1,950-2,600/month
\item \textbf{Implementation time:} 1-2 months
\end{itemize}

\item \textbf{City 1 Priority:}
\begin{itemize}
\item 100\% of City 1 shortage (13 MW) sourced externally
\item Recommend localized capacity addition or direct supply line
\item Expected savings: Rs. 13,000/month for full coverage
\end{itemize}

\item \textbf{External Supplier Negotiation:}
\begin{itemize}
\item Current rate: Rs. 1000/MW
\item Negotiate down to Rs. 750/MW $\Rightarrow$ Saves Rs. 3,250/month
\item Annual savings: Rs. 39,000 (7.8\% of external cost)
\item \textbf{Feasibility:} Moderate (requires volume guarantees)
\end{itemize}
\end{enumerate}

---

\section{Sensitivity Analysis}

\subsection{Scenario 1: 10\% Cost Increase on Low-Cost Route}

\textbf{Change:} Factory A → Market Y cost increases from Rs. 3 to Rs. 3.30/ton

\textbf{Impact Calculation:}
\begin{align}
\text{Change in route cost:} &\quad 0.30 \text{ Rs./ton}\\
\text{Units allocated:} &\quad 7 \text{ tons}\\
\text{Total cost increase:} &\quad 7 \times 0.30 = 2.10 \text{ Rs.}\\
\text{New total cost:} &\quad 153 + 2.10 = 155.10 \text{ Rs.}
\end{align}

\textbf{Allocation Stability:} Remains optimal (this was alternative optimal solution with $\Delta_{AY} = 0$)

\subsection{Scenario 2: 25\% Demand Increase (vs. 20\%)}

\textbf{New Demand:}
\begin{align}
\text{City 1:} \quad 30 \times 1.25 &= 37.5 \text{ MW}\\
\text{City 2:} \quad 35 \times 1.25 &= 43.75 \text{ MW}\\
\text{City 3:} \quad 25 \times 1.25 &= 31.25 \text{ MW}\\
\text{Total:} \quad &= 112.5 \text{ MW}
\end{align}

\textbf{New Shortage:}
\begin{equation}
112.5 - 95 = 17.5 \text{ MW}
\end{equation}

\textbf{Cost Impact:}
\begin{align}
\text{Additional external MW:} &\quad 17.5 - 13 = 4.5 \text{ MW}\\
\text{Additional external cost:} &\quad 4.5 \times 1000 = 4,500 \text{ Rs.}\\
\text{New total cost:} &\quad 49,710 + 4,500 = 54,210 \text{ Rs.}
\end{align}

---

\section{Summary Comparison of All Three Problems}

\begin{table}[h]
\centering
\caption{Comparative Analysis - Problem 1, 2, and 3}
\begin{tabular}{|c|c|c|c|}
\hline
\textbf{Metric} & \textbf{Problem 1} & \textbf{Problem 2} & \textbf{Problem 3} \\
\hline
Problem Type & Balanced & Demand > Supply & Demand > Supply \\
Total Supply & 37 units & 3,700 units & 95 MW \\
Total Demand & 37 units & 4,700 units & 108 MW \\
Imbalance & None & 1,000 units (27\%) & 13 MW (12\%) \\
\hline
Total Cost & 153 Rs. & 1,425,200 Rs. & 49,710 Rs. \\
Premium/External & None & 1,000,000 Rs. & 13,000 Rs. \\
Premium as \% & - & 70.2\% & 26.15\% \\
\hline
Critical Issue & None & Severe shortage & Moderate shortage \\
Routes Active & 5 & 12 & 8 \\
Dummy Usage & No & Yes (External) & Yes (External) \\
\hline
Key Strategy & Cost optimization & Shortage handling & Premium minimization \\
Recommendation & Maintain operations & Capacity expansion & Demand management \\
\hline
\end{tabular}
\end{table}

\subsection{Insights}

\begin{itemize}
\item \textbf{Problem 1:} Optimal resource allocation with balanced supply-demand
\item \textbf{Problem 2:} High-impact shortage scenario requiring significant external sourcing
\item \textbf{Problem 3:} Moderate shortage manageable through strategic planning
\item \textbf{Scale:} Problem 2 costs highest due to large shortage percentage and premium rates
\item \textbf{Solutions:} Each requires different intervention (capacity, demand reduction, supplier negotiation)
\end{itemize}

\newpage

% ============================================================================
% REFERENCES
% ============================================================================

\chapter*{References}

\begin{enumerate}

\item Taha, H. A. (2016). \textit{Operations Research: An Introduction} (10th ed.). Pearson Education.

\item Winston, W. L. (2004). \textit{Operations Research: Applications and Algorithms} (4th ed.). Cengage Learning.

\item Hillier, F. S., \& Lieberman, G. J. (2015). \textit{Introduction to Operations Research} (10th ed.). McGraw-Hill.

\item Sharma, S. D. (2009). \textit{Operations Research: Theory and Applications}. Macmillan India Limited.

\item Kalavathy, S. (2013). \textit{Operations Research} (4th ed.). Vikas Publishing House.

\item Mittal, K. C., \& Mohan, C. (2005). \textit{Optimization Methods in Operations Research and Systems Analysis}. New Age International.

\item Ravindran, A., Ragsdell, K. M., \& Reklaitis, G. V. (2006). \textit{Engineering Optimization: Methods and Applications} (2nd ed.). John Wiley \& Sons.

\item PuLP: Optimization Modelling in Python. Retrieved from https://github.com/coin-or/pulp

\item Cantú-Sifuentes, M., \& García-Díaz, A. (2012). Transportation Problem: Historical Algorithms and Recent Advances. In \textit{Studies in Computational Intelligence}.

\item Indian Institute of Technology (2024). Lecture Notes on Linear Programming and Transportation Problems.

\end{enumerate}

\end{document}